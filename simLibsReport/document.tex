\documentclass[12pt,a4paper]{article}

% ====== Packages ======
\usepackage[utf8]{inputenc}
\usepackage{amsmath, amssymb}
\usepackage{geometry}
\usepackage{setspace}
\usepackage{graphicx}
\usepackage{booktabs}
\usepackage{hyperref}
\usepackage{tikz}
\usetikzlibrary{arrows.meta, positioning, shapes.geometric}


% ====== Page setup ======
\geometry{margin=1in}
\setstretch{1.2}

% ====== Title ======
\title{Summary Report: simLIBS Python Library}
\author{Ehsan Namjoo}
\date{\today}

\begin{document}
	
	\maketitle
	\tableofcontents
	\newpage
	
	\section{Introduction}
	\textbf{simLIBS} (SimulatedLIBS) is a Python package for simulating Laser-Induced Breakdown Spectroscopy (LIBS) spectra using spectral data from the NIST LIBS database.  
	It is designed for:
	\begin{itemize}
		\item Generating synthetic datasets for machine learning,
		\item Educational demonstrations of LIBS principles,
		\item Research on plasma spectroscopy.
	\end{itemize}
	
	\section{Installation and Compatibility}
	\begin{itemize}
		\item Install with:
		\begin{verbatim}
			pip install simLIBS
		\end{verbatim}
		\item Compatible with Python 3.8 and above.
		\item Open-source under the MIT License.
	\end{itemize}
	
	\section{Core Features}
	\subsection{Simulation Initialization}
	The main class \texttt{SimulatedLIBS} takes parameters such as:
	\begin{itemize}
		\item \texttt{Te}: Electron temperature (eV),
		\item \texttt{Ne}: Electron density (cm$^{-3}$),
		\item \texttt{elements}: List of chemical symbols,
		\item \texttt{percentages}: Atomic composition percentages,
		\item \texttt{resolution}: Number of spectral points,
		\item \texttt{low\_w}, \texttt{upper\_w}: Wavelength range (nm),
		\item \texttt{max\_ion\_charge}: Maximum ionization state,
		\item \texttt{webscraping}: `'static'` or `'dynamic'` retrieval mode.
	\end{itemize}
	
	\subsection{Key Methods}
	\begin{itemize}
		\item \verb|plot(color, title)|: Plot the simulated spectrum.
		\item \verb|get_raw_spectrum()|: Return NIST raw spectral data.
		\item \verb|get_interpolated_spectrum()|: Cubic spline-interpolated spectrum.
		\item \verb|save_to_csv(filename)|: Export data to CSV.
		\item \verb|plot_ion_spectra()|: Display spectra for different ionization stages.
	\end{itemize}
	
	\section{Workflow Diagram}
	Figure~\ref{fig:workflow} shows the typical workflow from parameter selection to spectrum generation in \textbf{simLIBS}.
	

	
	\section{Dataset Generation}
	The package provides a \texttt{create\_dataset} method to generate large synthetic datasets from a composition table, varying plasma parameters (\texttt{Te} and \texttt{Ne}) within user-defined ranges.
	
	Example input table:
	
	\begin{center}
		\begin{tabular}{cccc}
			\toprule
			W & H & He & name \\
			\midrule
			50 & 25 & 25 & A \\
			30 & 60 & 10 & B \\
			40 & 40 & 20 & C \\
			\bottomrule
		\end{tabular}
	\end{center}
	
	\section{Educational and Visualization Tools}
	simLIBS supports animated visualizations of how LIBS spectra change with:
	\begin{itemize}
		\item Spectral resolution,
		\item Electron temperature (\texttt{Te}),
		\item Electron density (\texttt{Ne}).
	\end{itemize}
	
	\section{References and Applications}
	\begin{itemize}
		\item Developed by Marcin Kastek (IPPLM).
		\item Applied in LIBS research:
		\begin{itemize}
			\item M. Kastek et al., \textit{Spectrochimica Acta Part B} (2022).
			\item Chen et al., \textit{Optics Letters} (2023).
		\end{itemize}
	\end{itemize}
	
	\section{Summary Table}
	\begin{center}
		\begin{tabular}{@{}ll@{}}
			\toprule
			\textbf{Feature} & \textbf{Description} \\
			\midrule
			Purpose & Simulate LIBS spectra using NIST data \\
			Install & \verb|pip install simLIBS| \\
			Inputs & Te, Ne, elements, percentages, wavelength range, resolution \\
			Outputs & Plots, raw/interpolated spectra, CSV files \\
			Dataset Tool & \verb|create_dataset| for bulk generation \\
			Educational Use & Animations for teaching \\
			License & MIT, Python $\geq$ 3.8 \\
			\bottomrule
		\end{tabular}
	\end{center}
	
\end{document}
