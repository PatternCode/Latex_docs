\documentclass[12pt,a4paper]{article}

% ====== Packages ======
\usepackage[utf8]{inputenc}
\usepackage{amsmath, amssymb}
\usepackage{geometry}
\usepackage{setspace}
\usepackage{graphicx}
\usepackage{booktabs}
%\usepackage{hyperref}   % keep the boxes around numbers
\usepackage{tikz}
\usetikzlibrary{arrows.meta, positioning, shapes.geometric}
\usepackage[numbers]{natbib}  % For bibliography
\usepackage{graphicx,booktabs}    % Figures
\usepackage{enumitem}
\usepackage{url}
\usepackage[hidelinks]{hyperref}    %hidelinks romoves boxes



% ====== Page setup ======
\geometry{margin=1in}
\setstretch{1.2}

% ====== Title ======
\title{Summary Report: simLIBS Python Library to generate LIBS specrums}
\author{Ehsan Namjoo}
\date{\today}

\begin{document}
	
	\maketitle
	\tableofcontents
	\newpage
	\section{Introduction}
	Simulated Laser-induced Breakdown Spectroscopy (LIBS) spectra play a crucial role in research and development by providing a cost-effective and flexible alternative to experimental measurements. The simLIBS Python package is a useful tool designed to simulate LIBS spectra based on input plasma parameters and elemental compositions. It leverages atomic data and spectral libraries to generate realistic spectral profiles over specified wavelength ranges and resolutions.
	
	In this report, the simLIBS package is introduced, and its application in generating LIBS spectra for different steel grades based on their elemental compositions is demonstrated. Compositions are sampled within typical alloying ranges, with the iron content adjusted accordingly. The results show that even when concentrations vary widely within the allowable range for a given grade, the resulting spectra are not significantly altered. Furthermore, only minor differences are observed between the LIBS spectra of steel grades from similar families. It is also demonstrated that plasma parameters—namely plasma temperature and electron density—have a significant effect on the resulting spectra. The report concludes by outlining the conditions under which LIBS spectral datasets should be generated, as well as the potential design of machine learning algorithms for LIBS-based signal classification and concentration estimation. 
	
	\section{simLIBS package}
	\textbf{simLIBS} (Simulated LIBS) is a Python package designed for simulating LIBS spectra by utilizing spectral data from the NIST LIBS database \cite{NIST_LIBS}. Comprehensive documentation and details about the package can be found in \cite{simLIBS2024}. The package can be easily installed using the following command:
	
	\vspace{-0.5em}
	\begin{verbatim}
		pip install simLIBS
	\end{verbatim}
	\vspace{-1.5em}
	
		It is compatible with python 3.8 and above. The following command creates a simulated LIBS spectrum using the \texttt{SimulatedLIBS} class by specifying plasma conditions, elemental composition, and spectral parameters:
		\vspace{-0.5em}
		\begin{verbatim}
			libs = SimulatedLIBS(Te=1.0,
			Ne=10^{17},
			elements=['W','Fe','Mo'],
			percentages=[50,25,25],
			resolution=1000,
			low_w=200,
			upper_w=1000,
			max_ion_charge=3,
			webscraping='static')
		\end{verbatim}
		\vspace{-1.5em}
		
		Here, \texttt{Te=1.0} sets the electron temperature in electronvolts (eV). The electron temperature, controls the distribution of electrons among energy levels. Higher \texttt{Te} values correspond to hotter plasmas, leading to greater excitation and ionization of atoms, which significantly influences both the intensity and presence of spectral lines. \texttt{Ne=10\textasciicircum17} defines the electron density in cm\textsuperscript{-3}. The electron density  determines the number of free electrons in the plasma volume. The \texttt{elements} parameter specifies the elements included in the plasma, in this case, tungsten (W), iron (Fe), and molybdenum (Mo). Their relative atomic percentages are given by \texttt{percentages=[50,25,25]}, which sum to 100\%.
		
		The \texttt{resolution=1000} parameter controls the number of spectral data points, which indicates the number of discrete spectral points calculated over the chosen wavelength interval; higher resolution yields finer detail. The wavelength range of the simulation is set from 200 nm to 1000 nm using \texttt{low\_w=200} and \texttt{upper\_w=1000}. The maximum ionization charge state considered for spectral lines is limited by \texttt{max\_ion\_charge=3}. \texttt{max\_ion\_charge} parameter restricts the calculation to a maximum ionization stage (e.g., 2+, 3+), which can be useful to reduce computational load or to exclude highly ionized species unlikely to occur under given plasma conditions. Finally, \texttt{webscraping='static'} indicates that pre-cached atomic data is used for the simulation, avoiding live data scraping.
		
		The simulated LIBS spectrum can be visualized directly using the \texttt{plot} method. For example, the command
		
		\vspace{-0.5em}
		\begin{verbatim}
			libs.plot(color='blue', title='W Fe Mo composition')
		\end{verbatim}
		\vspace{-1.5em}
		plots the spectrum in blue and assigns the title ``W Fe Mo composition'' to the figure. Figure \ref*{fig:libs-spectrum-FeWMo} show the resultant plot.
		\begin{figure}[h!]
			\centering
			\includegraphics[width=0.8\textwidth]{FeWMo.pdf}
			\caption{LIBS spectrum for simulated sample.}
			\label{fig:libs-spectrum-FeWMo}
		\end{figure}
		
		The \texttt{simLIBS} package provides several key methods, in addition to the \texttt{plot(color, title)} method, for working with simulated spectra. 
		The \texttt{get\_raw\_spectrum()} method returns the raw spectral data obtained from the NIST atomic 
		databases, whereas the \texttt{get\_interpolated\_spectrum()} method generates a cubic spline--interpolated 
		version of the spectrum for smoother visualization and analysis. 
		For exporting results, the \texttt{save\_to\_csv(filename)} method enables saving the spectral data 
		in CSV format for external processing. 
		Finally, the \texttt{plot\_ion\_spectra()} method facilitates the visualization of spectra corresponding 
		to different ionization stages of the elements included in the simulation.
		
	\section{Simulation results}
	In this section, several simulations using the \texttt{simLIBS} package are conducted to investigate the effects of variations in elemental concentrations and plasma conditions. For the first experiment, two composition sets were selected to represent typical values found in industrial AISI~4340 samples. These two sets, referred to as the \emph{minimum} and \emph{maximum} sets, contain concentrations according to the allowable range of the AISI~4340 grade. The minimum set contains the lower limits of each constituent element, while the maximum set contains the upper limits. For both sets, the iron content was calculated as the remainder required to reach a total composition of approximately $100\%$. In addition to iron, eight elements commonly found in the alloy were considered: carbon (C), manganese (Mn), molybdenum (Mo), phosphorus (P), silicon (Si), sulfur (S), nickel (Ni), and chromium (Cr). Table \ref{tab:astm4340} show the concentration ranges of the alloying elements found in AISI~4340.
	\begin{table}[htbp]
		\centering
		\caption{AISI 4340 Steel – Chemical Composition}
		\label{tab:astm4340}
		\resizebox{\textwidth}{!}{%
			\begin{tabular}{@{} llcccccccc @{}} 
				\toprule
				\textbf{Standard} & \textbf{Grade} & \textbf{C (\%)} & \textbf{Mn (\%)} & \textbf{P (\%)} & \textbf{S (\%)} & \textbf{Si (\%)} & \textbf{Ni (\%)} & \textbf{Cr (\%)} & \textbf{Mo (\%)} \\
				\midrule
				SAE/AISI & 4340 & 0.38--0.43 & 0.60--0.80 & $\leq$0.035 & $\leq$0.040 & 0.15--0.35 & 1.65--2.00 & 0.70--0.90 & 0.20--0.30 \\
				\bottomrule
			\end{tabular}%
		}
	\end{table}
	
	
	LIBS spectra were generated for both composition sets under identical plasma parameters, with the electron temperature fixed at 1.0~eV and the electron density at $1 \times 10^{17}~\text{cm}^{-3}$. A spectral range from 200 to 1000~nm was used, with a resolution of 1000. Ionization states up to the third charge were included in the modeling. The resulting spectra plotted together for both concentration sets is depicted in Figure \ref{fig:min-max}.
	
	
	
	\begin{figure}[h!]
		\centering
		\includegraphics[width=0.8\textwidth]{min-max.pdf}
		\caption{Simulated LIBS spectra of AISI 4240 with different sompositions (min and max ranges).}
		\label{fig:min-max}
	\end{figure} 
	The spectra corresponding to the different composition sets appear to be nearly indistinguishable, effectively overlapping one another. This observation indicates that variations in elemental concentrations within the predefined compositional ranges for AISI 4340 do not significantly affect the overall LIBS spectral profile. In other words, small to moderate changes in alloy composition within these limits result in LIBS spectra that are largely similar—indeed, in many cases, effectively indistinguishable.
	
%	For the next experiment, under the same conditions as in the earlier experiment, the spectra of two different alloy steel grades, AISI~4020 and AISI~4140, were compared (see Figure~\ref{fig:4020-4140}). both ot the grades belonged to 4XXX family according to AISI standard. however AISI 4020 is a member of 40XX group which are known as Molybdenum steel, while AISI 4140 is a member of 41xx grouop which as generally known as Chromium-Molybdenum Steel. The compositions for each grade were sampled uniformly within the predefined range for that grade. The results showed that, in this case as well, no significant differences were observed. Interestingly, the resulting spectra were very similar to those obtained in the earlier experiment for AISI~4340. The shapes of the spectra from both experiments are nearly identical; the only noticeable difference is that the peaks in the second experiment are slightly higher. This difference could be attributed to the higher percentage of iron, the predominant constituent, in the second experiment.
	
	
	For the next experiment, conducted under the same conditions as the earlier one, the spectra of two different alloy steel grades (AISI~4020 and AISI~4140) were compared (See Figure~\ref{fig:4020-4140}). Both grades belong to the 4xxx family according to the AISI standard. However, AISI~4020 is part of the 40xx group, commonly referred to as \emph{molybdenum steels}, whereas AISI~4140 belongs to the 41xx group, generally known as \emph{chromium--molybdenum steels} \cite{runsomSAEgrades}. The compositions of each grade were sampled uniformly within the predefined range specified for that grade.
	
	The results again revealed no significant differences. Interestingly, the spectra obtained were highly similar to one another and closely resembled those from the earlier experiment with AISI 4340. The overall spectral shapes from both experiments are nearly identical; the only noticeable distinction is that the peaks in the second experiment appear slightly higher. This increase is likely due to the higher proportion of iron, the predominant constituent, present in the second experiment.
	
	The objective of these two recent experiments was to demonstrate that the LIBS spectrum of a given steel grade may remain largely unchanged, even when elemental concentrations vary substantially within the permissible range. Moreover, the LIBS spectra of different steel grades from the same family, such as AISI 4020 and AISI 4140, can also appear very similar.
	
	
	\begin{figure}[h!]
		\centering
		\includegraphics[width=0.8\textwidth]{4020-4140.pdf}
		\caption{Simulated LIBS spectra of AISI 4020 vs 4140.}
		\label{fig:4020-4140}
	\end{figure} 

 	
 	As the third experiment, conducted under the same conditions, the spectra of four different steel grades (AISI~4140, AISI~304, AISI~1095, and AISI~4340) were compared (see Figure~\ref{fig:4041-304-1095-4340}).  
 	
 	All four grades display broadly similar spectral profiles, characterised by pronounced peaks in the 220--280~nm region and markedly lower intensities beyond 300~nm. This general peak pattern is consistent with the results of the earlier experiments, reflecting the dominant contribution of iron to the LIBS signal in all cases. The spectra of AISI~4140, AISI~1095, and AISI~4340 are nearly identical in both shape and absolute intensity, suggesting very similar overall chemical compositions, with iron as the primary constituent and only minor variations in alloying elements such as carbon, manganese, and molybdenum.  
 	
 	In contrast, the spectrum of AISI~304 exhibits subtle but noticeable deviations from the others. In particular, the relative intensities of the major peaks near 230~nm and 260~nm are slightly different, and the fine structure in the 220--240~nm range shows small variations in peak prominence. These differences are related with AISI~304’s significantly higher chromium content (\(\sim18\%\)) and substantial nickel content (\(\sim8\%\)), compared with the predominantly low-alloy compositions of the other grades. 
 	
   
 	
 	Overall, while the LIBS spectra of high-alloy stainless steels such as AISI~304 still share the general iron-dominated profile of carbon and low-alloy steels, the relative peak intensities and subtle spectral variations arising from alloying elements can provide useful distinguishing features when analysed carefully.
 	
	
	\begin{figure}[h!]
		\centering
		\includegraphics[width=0.8\textwidth]{4041-304-1095-4340.pdf}
		\caption{Simulated LIBS spectra of four different steel grades.}
		\label{fig:4041-304-1095-4340}
	\end{figure} 
	

	For the next experiment, the effect of electron temperature ($T_e$) on the LIBS spectrum was investigated. To this end, three different values of electron temperature (1.0~eV, 0.8~eV, and 0.6~eV) were considered for generating the LIBS spectrum of the AISI~4140 steel grade (see Figure~\ref{fig:Effect of T_e}).  
	
	According to Figure~\ref{fig:Effect of T_e}, in the low-wavelength region (200--350~nm), the spectrum for $T_e = 1.0$~eV exhibits the most prominent peaks. The $T_e = 0.8$~eV curve shows moderately strong peaks, while the $T_e = 0.6$~eV curve displays the weakest intensities. This pattern indicates that higher electron temperatures significantly increase the population of high-energy excited states, thereby enhancing the intensity of high-energy transitions.  
	
	However, the intensity enhancement does not follow a consistent trend across the entire wavelength range. In the intermediate-wavelength region (350--450~nm), the spectra corresponding to lower electron temperatures ($T_e$) exhibit stronger peaks, and the differences between the three curves become less pronounced. This suggests that lower $T_e$ values may favour transitions from lower-energy excited states, which are more heavily populated at reduced temperatures according to the Boltzmann distribution. The Boltzmann distribution describes how, in a system at thermal equilibrium, the number of particles $N_j$ in an excited state $j$ with energy $E_j$ is related to the number in the ground state $N_0$ by  
	\[
	\frac{N_j}{N_0} = \frac{g_j}{g_0} \exp\left(-\frac{E_j - E_0}{k_\mathrm{B} T_e}\right),
	\]  
	where $g_j$ and $g_0$ are the statistical weights of the respective states, $k_\mathrm{B}$ is the Boltzmann constant, and $T_e$ is the electron temperature \cite{Cremers2013LIBSHandbook}. As $T_e$ decreases, the exponential term strongly favours lower-energy states, increasing their relative populations. Also the  energy of a photon emitted during a transition is given by  
	\[
	E_{\text{photon}} = h\nu = \frac{hc}{\lambda},
	\]  
	where \(h\) is Planck’s constant (\(6.626\times10^{-34}~\mathrm{J\,s}\)), \(c\) is the speed of light in vacuum (\(3.00\times10^8~\mathrm{m/s}\)), \(\nu\) is the photon frequency, and \(\lambda\) is its wavelength. A smaller energy gap between the initial and final states (\(E_{\text{photon}}\) low) corresponds to a longer wavelength (\(\lambda\) large). Consequently, transitions from lower-energy states tend to occur at longer wavelengths.  
	
	At high electron temperatures, although lower-energy states are still populated, the distribution of electrons spreads over many higher-energy states, reducing the relative population advantage of the low-energy states. This means their corresponding long-wavelength lines do not experience the same relative enhancement as at lower $T_e$. In contrast, at reduced $T_e$, most atoms remain in the ground or low-lying excited states, making the associated longer-wavelength transitions proportionally stronger in the observed spectrum.  
	
	It is also observed that changing the electron temperature affects the overall curve patterns and may appear to shift the positions of certain peaks. Nevertheless, since the wavelengths associated with specific elements are intrinsic and fixed, any apparent peak shifts are more likely the result of changes in relative intensities. Such intensity variations can lead to misinterpretation in LIBS spectrum analysis.
	
		
		
	\begin{figure}[h!]
		\centering
		\includegraphics[width=0.8\textwidth]{Te.pdf}
		\caption{Effect of $T_e$ on AISI~4140 spectrum.}
		\label{fig:Effect of T_e}
	\end{figure}
	
	For the final experiment, the effect of varying the electron density on the LIBS spectrum was investigated. The experimental conditions are the same as in the previous experiment, except that the electron temperature is fixed at $T_e = 1.0$~eV, while three electron density values are considered: $1\times10^{16}$~cm$^{-3}$, $1\times10^{17}$~cm$^{-3}$, and $1\times10^{18}$~cm$^{-3}$. Figure~\ref{fig:Effect of N_e} illustrates the results for this case.
	
	 As it is observed from the Figure \ref{fig:Effect of N_e}, in the low-wavelength region, all three spectra exhibit strong peaks. The spectrum for $N_e = 1\times 10^{16}$~cm$^{-3}$ shows slightly higher peak intensities than the others. It seems that for the low-wavelength region, the $N_e$ has a inverse effect on the intensities, i.e., spectrum for lower $N_e$ show stronger intensities. This behavior is completly different in the intermediate-wavelength range (300--450~nm), where the $N_e = 1\times 10^{16}$~cm$^{-3}$ case shows more  smaller peaks compared to the other densities.
	 
	 
	 
	
		
	
	\begin{figure}[h!]
		\centering
		\includegraphics[width=0.8\textwidth]{Ne.pdf}
		\caption{Effect of $N_e$ on AISI~4140 spectrum.}
		\label{fig:Effect of N_e}
	\end{figure}

	
	\section{Discussion}
	According to the observations presented in the previous section, this section focuses on discussing the findings and highlighting points that appear to be important in the qualitative and quantitative analysis of LIBS spectra in the context of steel grade classification and concentration estimation.
	
	As observed in Figure~\ref{fig:min-max}, the spectra acquired from the same steel grade but with two different concentrations of alloying elements showed no noticeable differences. It was also found that the LIBS spectra of two different steel grades within the same category are highly similar. As a general rule, if there is no significant variation in the observed patterns (in this case, the LIBS spectra), machine learning models designed to recognize and classify such patterns are unlikely to perform well. Moreover, training a model on a dataset with minimal variation can easily lead to overfitting. In contrast, differences in elemental concentrations become more apparent in the corresponding LIBS spectra when the steel grades belong to distinctly different categories, such as carbon steel versus stainless steel. In other words, simply varying elemental concentrations within their predefined ranges may not produce a dataset sufficiently diverse to capture the true statistical characteristics of a given steel grade. This is especially true when other factors, such as electron temperature and density, can influence the spectral profile more effectively and noticeably than changes in elemental concentration alone. Therefore, creating a dataset solely by altering concentrations within the allowed range is not a reliable strategy for providing a robust dataset to train machine learning algorithms for classification or concentration estimation.
	
	
	There are many parameters that influence the LIBS spectrum in practice. However, in \texttt{simLIBS}, the user has access to only a limited subset of these parameters. In reality, factors such as electron temperature ($T_e$) and electron density ($n_e$) cannot be directly controlled by the user, even though they are adjustable in \texttt{simLIBS}.
	  
	
	The characteristics of a LIBS spectrum—including the number of detected peaks, the absolute and relative intensities of atomic emission lines, the continuum background, and other spectral features—depend on numerous factors. These include laser pulse energy and duration, focusing conditions, the detection system, and acquisition parameters (e.g., integration time and delay time). 	  
	When comparing spectra obtained with different LIBS systems or under varying experimental conditions, these factors must be taken into account. Even when analyzing the same sample, variations in the experimental setup can lead to spectra that are similar but not identical \cite{AppliedPhotonics2019LIBSQualitative}.
	
	Another important point in LIBS spectrum analysis is that each element has well-established and well-documented spectral specifications. Every element exhibits a unique wavelength “fingerprint,” characterized by specific peaks and spectral lines within defined ranges. By leveraging this knowledge, spectrum analysis can be focused on the elements of interest and their corresponding wavelength bands. Incorporating prior knowledge of the intrinsic and constant properties of the constituent elements in steel grades can support the development of models that more accurately capture the realistic behavior of LIBS spectra. 
	
	 
	 
	
	\section{Conlusion}
	According to the material mentioned in the previous sections, the following considerations seem to be very beneficial for developing capable models for analyzing LIBS spectra for steel grade classification and concentration estimation.
%	\begin{itemize}
%		\item \textbf{Problem statement:} Utilizing LIBS spectrum for analysins steel grades generally is perform to aim two different problems: classification, and concentration estimation. in estimation scenario the peaks at specified waveleangths are measured and by comparing those with peake achives from sample references, the concentrations of elements is estimated. It is clear the physical situations and equipment setting should be fixed during acquiring the spectrum of refrence sample, and test samples. It is a well-known precedure that the intensities acquired by LISB are compared to actual percentations of  reference samples to find the relationship between them. this relationship usually is refered as calibration curves. in this scenario the amplitude of peak and the corresponding wavelenghs are very impoetaant to create a accurate estimatior. On the other hand and in contrast to estimation, in classification, the goal is to decide about the category that the sample belongs to. to this aim the general structure of a new sample is compared with those that are targeted by clasification method. this can be used by macchine learning algorithm to learn the structure and distinguishing characteristics of spectrum of each category and based on them decide about the class of the new sample. most probably the MEDALS project is aiming at classification. 
%		\item \textbf{Fixing the measurement parameters and using the same setting:} In eigher case (classification, or concentration estimation) the settins and parameters for creating LIBS specrum should be fixed. In reality even by considering fix setting, the same LIBS device do not generate exactly the same spectrum for the same sample in multiple measurements, as the nature of plasma and the phisics involed in acquiring signal has some level of interinsic uncertanty by its own. therefore fixing the setting in measurements is a fundamental requirement to insure that the intrinsic nature of the plasma and wavelength registration technique remain as the only source of uncertainty. In the case using similator like simLIBS, one can fix the parameters like, $T_e$, $N_e$, and ionisation states. Under the typical power densities employed in LIBS experiments (a few tenths of~GW/cm\textsuperscript{2}), spectra generally contain emission lines from neutral atoms (I) and singly ionised species (II). Higher ionisation states are not expected.and add a suitable source of noise that are capable of reflecting the real noise in LIBS spectrum. in the classification scenario altering the concentrations of sample could also be considered, however as mentioned in earlier sections it could not be enough. al
%		\item \textbf{State of nature:} the term “state of nature” refers to the true underlying situation or condition that affects outcomes but may not be directly observed \cite{Duda2001PatternClassification}. In either clasifcation, or estimation problem the data that is acquired to be fed into the model should be informative and contain discriminating informations. the actual range that need to be examined, or peak positions could be part of these information. in the case of LIBS spectrum for steal grade finding out the essential wavelength for elloying elements and some prior knowledge about the interinsic peaks of each element could be very beneficial in creating a reliable model.  
%	\end{itemize}
	
	 
	 \begin{itemize}
	 	\item \textbf{Problem statement:} Utilizing LIBS spectra for analyzing steel grades is generally performed to address two different problems: classification and concentration estimation. In the estimation scenario, the peaks at specified wavelengths are measured and, by comparing these with peaks obtained from reference samples, the concentrations of elements are estimated. It is clear that the physical conditions and equipment settings should be fixed during the acquisition of both reference and test sample spectra. A well-known procedure is to compare the intensities acquired by LIBS with the actual concentrations of the reference samples to find the relationship between them. This relationship is usually referred to as a calibration curve. In this scenario, the amplitude of the peak and the corresponding wavelengths are very important for creating an accurate estimator. On the other hand, in classification, the goal is to determine the category to which the sample belongs. To this aim, the general structure of a new sample is compared with those targeted by the classification method. Machine learning algorithms can be used to learn the structure and distinguishing characteristics of the spectra of each category and, based on this knowledge, decide on the class of a new sample. Most probably, the MEDALS project is aiming at classification. 
	 	
	 	\item \textbf{Fixing the measurement parameters and using the same settings:} In either case (classification or concentration estimation), the settings and parameters for creating LIBS spectra should be fixed. In practice, even under fixed settings, the same LIBS device may not generate exactly the same spectrum for the same sample in multiple measurements, since the nature of plasma and the physics involved in signal acquisition introduce some intrinsic uncertainty. Therefore, fixing the measurement settings is a fundamental requirement to ensure that the intrinsic properties of the plasma and the wavelength registration technique remain the only sources of uncertainty. In the case of using a simulator such as \textit{simLIBS}, one can fix parameters such as $T_e$, $N_e$, and ionization states. Under the typical power densities employed in LIBS experiments (a few tenths of~GW/cm\textsuperscript{2}), spectra generally contain emission lines from neutral atoms (I) and singly ionized species (II). Higher ionization states are not expected. It is also useful to add a suitable source of noise that reflects the real noise observed in LIBS spectra. In the classification scenario, altering the concentrations of the sample could also be considered; however, as mentioned in earlier sections, this may not be sufficient. 
	 	
	 	\item \textbf{State of nature:} The term \textit{state of nature} refers to the true underlying situation or condition that affects outcomes but may not be directly observed \cite{Duda2001PatternClassification}. In either classification or estimation problems, the data acquired to be fed into the model should be informative and contain discriminating information. The actual range that needs to be examined, or specific peak positions, could be part of this information. In the case of LIBS spectra for steel grading, identifying the essential wavelengths of alloying elements and having prior knowledge about the intrinsic peaks of each element could be very beneficial for creating a reliable model.
	 \end{itemize}
	 
	
	
	
	\section*{Appendix}
	\textbf{Boltzmann distribution:}
	From a physical perspective, the population of excited states follows the Boltzmann distribution:
	\[
	n_i \propto g_i \exp\left(-\frac{E_i}{k_B T_e}\right)
	\]
	where $E_i$ is the excitation energy of state $i$, $g_i$ is its degeneracy, and $k_B$ is Boltzmann’s constant. As $T_e$ increases, the population of higher-energy states rises, leading to stronger emissions at shorter wavelengths. Conversely, lower $T_e$ favors transitions from lower-energy states, which may appear relatively stronger due to the reduced population of higher-energy levels.
	
	The effect of electron temperature on line intensity is wavelength-dependent. High-energy transitions become more prominent at higher $T_e$, while some lower-energy transitions may be relatively stronger at lower $T_e$. The spectral response is thus governed by the excitation energy of each atomic transition.
	
	
	\textbf{Why temperature is measured in eV:}
	In plasma physics and spectroscopy, the concept of electron temperature ($T_e$) is frequently expressed in units of electron volts (eV) rather than in the conventional Kelvin (K). This practice arises from the fact that temperature at the microscopic scale is fundamentally a measure of the average kinetic energy of particles. According to kinetic theory, the mean kinetic energy of a particle is related to the temperature by  
	\[
	\langle E \rangle = \frac{3}{2} k_B T,
	\]  
	where $\langle E \rangle$ is the average kinetic energy per particle, $k_B$ is the Boltzmann constant, and $T$ is the absolute temperature in Kelvin. Since the energies of atomic and electronic transitions, as well as ionization potentials, are naturally expressed in electron volts, it is convenient to use the same unit for electron temperature.  
	
	The conversion between the two units follows from $E = \frac{3}{2}k_B T$, where $1~\text{eV} = 1.602 \times 10^{-19}~\text{J}$. This corresponds to approximately $11{,}600~\text{K}$. Thus, for example, an electron temperature of $T_e = 1.0$~eV corresponds to about $11{,}600$~K, while $T_e = 0.8$~eV and $T_e = 0.6$~eV correspond to roughly $9{,}300$~K and $7{,}000$~K, respectively.  
	
	Expressing electron temperature in eV is particularly useful in the context of Laser-Induced Breakdown Spectroscopy (LIBS). In such plasmas, electrons collide with atoms and ions to excite them, and the excitation and ionization energies involved are typically quoted in electron volts. By describing $T_e$ directly in eV, it becomes straightforward to compare the thermal energy of the electron population with the energy thresholds required for specific transitions. For example, if a spectral line of iron requires an excitation energy of about 3~eV, a plasma with $T_e = 1$~eV will only have a small fraction of electrons energetic enough to excite that transition, whereas a plasma with $T_e = 5$~eV will have a much larger fraction of electrons capable of doing so. This direct comparability makes electron volts the most natural and practical unit for electron temperature in plasma spectroscopy.
	
	
	
	
%	\textbf{Effect of $N_e$ variation at fixed $T_e$}
%	When the electron density ($N_e$) is varied at a fixed electron temperature ($T_e$), the UV region of the spectrum (approximately 200--300~nm) is most sensitive to density effects. At low $N_e$, the UV peaks are sharp and tall because there is little collisional de-excitation, minimal self-absorption, and low continuum emission. As $N_e$ increases, the line width grows due to Stark broadening,
%	\[
%	\Delta \lambda_{\rm Stark} \propto N_e,
%	\]
%	collisional quenching reduces the effective emission probability,
%	\[
%	A_{ij}^{\rm eff} = \frac{A_{ij}}{1 + C_u/A_{ij}}, \quad C_u = N_e q_u,
%	\]
%	and self-absorption flattens strong lines, with optical depth
%	\[
%	\tau \propto N_l f_{ij} L.
%	\]
%	The continuum background also rises roughly as $N_e^2$, further lowering line contrast. Together, these effects make the UV peaks tallest and sharpest at low electron densities, and progressively flattened and blended at higher densities.  
%	
%	In the intermediate-wavelength range (roughly 300--450~nm), the behavior is different. Increasing $N_e$ favors neutral atoms through recombination and modifies the ionisation balance,
%	\[
%	\frac{n_{i+1}}{n_i} \propto \frac{1}{N_e} e^{-\chi_i/k_B T_e},
%	\]
%	which enhances emission in this band. However, higher $N_e$ also increases Stark broadening, merging nearby lines. As a result, at low $N_e$, the spectrum shows many narrow, resolved peaks, while at high $N_e$, fewer peaks appear but the total emission intensity grows, producing broader, stronger spectral features. These complementary trends in the UV and visible bands explain the characteristic two-regime behavior observed in plasma spectra.
	
		% ---------- References ----------
	\newpage
	%	\bibliographystyle{plainnat}
	\bibliographystyle{unsrtnat}
	%	\bibliographystyle{abbrvnat}
	\bibliography{references}  % Create a file named references.bib
	

	
\end{document}
