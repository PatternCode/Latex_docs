\documentclass[12pt,a4paper]{article}

% ====== Packages ======
\usepackage[utf8]{inputenc}
\usepackage{amsmath, amssymb}
\usepackage{geometry}
\usepackage{setspace}
\usepackage{graphicx}
\usepackage{booktabs}
\usepackage{hyperref}
\usepackage{tikz}
\usetikzlibrary{arrows.meta, positioning, shapes.geometric}
\usepackage[numbers]{natbib}  % For bibliography
\usepackage{graphicx}    % Figures



% ====== Page setup ======
\geometry{margin=1in}
\setstretch{1.2}

% ====== Title ======
\title{Summary Report: simLIBS Python Library to generate LIBS specrums}
\author{Ehsan Namjoo}
\date{\today}

\begin{document}
	
	\maketitle
	\tableofcontents
	\newpage
	\section{Introduction}
	Simulated LIBS spectra play a crucial role in research and development by providing a cost-effective and flexible alternative to experimental measurements. The simLIBS Python package is a powerful tool designed to simulate LIBS spectra based on input plasma parameters and elemental compositions. It leverages atomic data and spectral libraries to generate realistic spectral profiles over specified wavelength ranges and resolutions.
	
	This report introduces the simLIBS package in detail and demonstrates how it can be used to generate LIBS spectra for different steel grades by specifying their elemental compositions. By sampling compositions within typical alloying ranges and balancing iron content accordingly, simLIBS enables the creation of diverse and representative spectral datasets. It is shown that even when selecting widely varying concentrations within the allowable range for a given grade, the resulting spectra do not change significantly. Additionally, it is observed that LIBS spectra of different steel grades from similar families exhibit only minor differences. The report concludes by raising questions regarding the conditions and usefulness of creating LIBS spectral datasets, as well as the structure of machine learning algorithms that could be utilized for LIBS signal classification and concentration estimation. 
	
	\section{simLIBS package}
	\textbf{simLIBS} (SimulatedLIBS) is a Python package for simulating Laser-Induced Breakdown Spectroscopy (LIBS) spectra using spectral data from the NIST LIBS database \cite{NIST_LIBS}. All information about this package is available in \cite{simLIBS2024}. The packeg can be easily installed by: 
	\vspace{-0.5em}
	\begin{verbatim}
		pip install simLIBS
	\end{verbatim}
	\vspace{-1.5em}
		It is compatible with python 3.8 and above. The following command creates a simulated LIBS spectrum using the \texttt{SimulatedLIBS} class by specifying plasma conditions, elemental composition, and spectral parameters:
		\vspace{-0.5em}
		\begin{verbatim}
			libs = SimulatedLIBS(Te=1.0,
			Ne=10^{17},
			elements=['W','Fe','Mo'],
			percentages=[50,25,25],
			resolution=1000,
			low_w=200,
			upper_w=1000,
			max_ion_charge=3,
			webscraping='static')
		\end{verbatim}
		\vspace{-1.5em}
		
		Here, \texttt{Te=1.0} sets the electron temperature in electronvolts (eV). The electron temperature, \texttt{Te}, given in electron volts (eV), controls the distribution of electrons among energy levels according to the Boltzmann and Saha equations. Higher \texttt{Te} values correspond to hotter plasmas, leading to greater excitation and ionization of atoms, which significantly influences both the intensity and presence of spectral lines. \texttt{Ne=10\textasciicircum17} defines the electron density in cm\textsuperscript{-3}. The electron density, \texttt{Ne},  determines the number of free electrons in the plasma volume; it plays a critical role in ionization equilibrium and in pressure broadening of spectral features, particularly through Stark broadening mechanisms. The \texttt{elements} parameter specifies the elements included in the plasma—in this case, tungsten (W), iron (Fe), and molybdenum (Mo). Their relative atomic percentages are given by \texttt{percentages=[50,25,25]}, which sum to 100\%.
		
		The \texttt{resolution=1000} parameter controls the number of spectral data points, which indicates the number of discrete spectral points calculated over the chosen wavelength interval; higher resolution yields finer detail but may increase computation time. The wavelength range of the simulation is set from 200 nm to 1000 nm using \texttt{low\_w=200} and \texttt{upper\_w=1000}. The maximum ionization charge state considered for spectral lines is limited by \texttt{max\_ion\_charge=3}. \texttt{max\_ion\_charge} parameter restricts the calculation to a maximum ionization stage (e.g., 2+, 3+), which can be useful to reduce computational load or to exclude highly ionized species unlikely to occur under given plasma conditions.Finally, \texttt{webscraping='static'} indicates that pre-cached atomic data is used for the simulation, avoiding live data scraping.
		
		The simulated LIBS spectrum can be visualized directly using the \texttt{plot} method. For example, the command
		
		\vspace{-0.5em}
		\begin{verbatim}
			libs.plot(color='blue', title='W Fe Mo composition')
		\end{verbatim}
		\vspace{-1.5em}
		plots the spectrum in blue and assigns the title ``W Fe Mo composition'' to the figure. Figure \ref*{fig:libs-spectrum-FeWMo} show the resultant plot.
		\begin{figure}[h!]
			\centering
			\includegraphics[width=0.8\textwidth]{FeWMo.pdf}
			\caption{LIBS spectrum for simulated sample.}
			\label{fig:libs-spectrum-FeWMo}
		\end{figure}
		
		The \texttt{simLIBS} package provides several key methods, in addition to the \texttt{plot(color, title)} method, for working with simulated spectra. 
		The \texttt{get\_raw\_spectrum()} method returns the raw spectral data obtained from the NIST atomic 
		databases, whereas the \texttt{get\_interpolated\_spectrum()} method generates a cubic spline--interpolated 
		version of the spectrum for smoother visualization and analysis. 
		For exporting results, the \texttt{save\_to\_csv(filename)} method enables saving the spectral data 
		in CSV format for external processing. 
		Finally, the \texttt{plot\_ion\_spectra()} method facilitates the visualization of spectra corresponding 
		to different ionization stages of the elements included in the simulation.
		
		
		
		
		


	\section{Workflow Diagram}
%	Figure~\ref{fig:workflow} shows the typical workflow from parameter selection to spectrum generation in \textbf{simLIBS}.
	

	
	\section{Dataset Generation}
	The package provides a \texttt{create\_dataset} method to generate large synthetic datasets from a composition table, varying plasma parameters (\texttt{Te} and \texttt{Ne}) within user-defined ranges.
	
	Example input table:
	
	\begin{center}
		\begin{tabular}{cccc}
			\toprule
			W & H & He & name \\
			\midrule
			50 & 25 & 25 & A \\
			30 & 60 & 10 & B \\
			40 & 40 & 20 & C \\
			\bottomrule
		\end{tabular}
	\end{center}
	
	\section{Educational and Visualization Tools}
	simLIBS supports animated visualizations of how LIBS spectra change with:
	\begin{itemize}
		\item Spectral resolution,
		\item Electron temperature (\texttt{Te}),
		\item Electron density (\texttt{Ne}).
	\end{itemize}
	
	\section{References and Applications}
	\begin{itemize}
		\item Developed by Marcin Kastek (IPPLM).
		\item Applied in LIBS research:
		\begin{itemize}
			\item M. Kastek et al., \textit{Spectrochimica Acta Part B} (2022).
			\item Chen et al., \textit{Optics Letters} (2023).
		\end{itemize}
	\end{itemize}
	
	\section{Summary Table}
	\begin{center}
		\begin{tabular}{@{}ll@{}}
			\toprule
			\textbf{Feature} & \textbf{Description} \\
			\midrule
			Purpose & Simulate LIBS spectra using NIST data \\
			Install & \verb|pip install simLIBS| \\
			Inputs & Te, Ne, elements, percentages, wavelength range, resolution \\
			Outputs & Plots, raw/interpolated spectra, CSV files \\
			Dataset Tool & \verb|create_dataset| for bulk generation \\
			Educational Use & Animations for teaching \\
			License & MIT, Python $\geq$ 3.8 \\
			\bottomrule
		\end{tabular}
	\end{center}
	
		% ---------- References ----------
	\newpage
	%	\bibliographystyle{plainnat}
	\bibliographystyle{unsrtnat}
	%	\bibliographystyle{abbrvnat}
	\bibliography{references}  % Create a file named references.bib
	

	
\end{document}
