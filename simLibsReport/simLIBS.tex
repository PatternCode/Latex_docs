\documentclass[12pt,a4paper]{article}

% ====== Packages ======
\usepackage[utf8]{inputenc}
\usepackage{amsmath, amssymb}
\usepackage{geometry}
\usepackage{setspace}
\usepackage{graphicx}
\usepackage{booktabs}
\usepackage{hyperref}
\usepackage{tikz}
\usetikzlibrary{arrows.meta, positioning, shapes.geometric}
\usepackage[numbers]{natbib}  % For bibliography
\usepackage{graphicx,booktabs}    % Figures



% ====== Page setup ======
\geometry{margin=1in}
\setstretch{1.2}

% ====== Title ======
\title{Summary Report: simLIBS Python Library to generate LIBS specrums}
\author{Ehsan Namjoo}
\date{\today}

\begin{document}
	
	\maketitle
	\tableofcontents
	\newpage
	\section{Introduction}
	Simulated Laser-induced Breakdown Spectroscopy (LIBS) spectra play a crucial role in research and development by providing a cost-effective and flexible alternative to experimental measurements. The simLIBS Python package is a useful tool designed to simulate LIBS spectra based on input plasma parameters and elemental compositions. It leverages atomic data and spectral libraries to generate realistic spectral profiles over specified wavelength ranges and resolutions.
	
	In this report, the simLIBS package is introduced in detail, and its use in generating LIBS spectra for different steel grades by specifying their elemental compositions is demonstrated. Compositions are sampled within typical alloying ranges, and iron content is balanced accordingly. It is shown that even when concentrations varying widely within the allowable range for a given grade are selected, the resulting spectra are not significantly altered. Additionally, only minor differences are exhibited between LIBS spectra of different steel grades from similar families. It is also demonstrated that plasma physical parameters, plasma temperature and electron density, have a significant effect on the resulting spectra. The report concludes by raising questions regarding the conditions and usefulness of creating LIBS spectral datasets, as well as the structure of machine learning algorithms that could be employed for LIBS signal classification and concentration estimation. 
	
	\section{simLIBS package}
	\textbf{simLIBS} (Simulated LIBS) is a Python package designed for simulating LIBS spectra by utilizing spectral data from the NIST LIBS database \cite{NIST_LIBS}. Comprehensive documentation and details about the package can be found in \cite{simLIBS2024}. The package can be easily installed using the following command:
	
	\vspace{-0.5em}
	\begin{verbatim}
		pip install simLIBS
	\end{verbatim}
	\vspace{-1.5em}
	
		It is compatible with python 3.8 and above. The following command creates a simulated LIBS spectrum using the \texttt{SimulatedLIBS} class by specifying plasma conditions, elemental composition, and spectral parameters:
		\vspace{-0.5em}
		\begin{verbatim}
			libs = SimulatedLIBS(Te=1.0,
			Ne=10^{17},
			elements=['W','Fe','Mo'],
			percentages=[50,25,25],
			resolution=1000,
			low_w=200,
			upper_w=1000,
			max_ion_charge=3,
			webscraping='static')
		\end{verbatim}
		\vspace{-1.5em}
		
		Here, \texttt{Te=1.0} sets the electron temperature in electronvolts (eV). The electron temperature, controls the distribution of electrons among energy levels. Higher \texttt{Te} values correspond to hotter plasmas, leading to greater excitation and ionization of atoms, which significantly influences both the intensity and presence of spectral lines. \texttt{Ne=10\textasciicircum17} defines the electron density in cm\textsuperscript{-3}. The electron density  determines the number of free electrons in the plasma volume. The \texttt{elements} parameter specifies the elements included in the plasma, in this case, tungsten (W), iron (Fe), and molybdenum (Mo). Their relative atomic percentages are given by \texttt{percentages=[50,25,25]}, which sum to 100\%.
		
		The \texttt{resolution=1000} parameter controls the number of spectral data points, which indicates the number of discrete spectral points calculated over the chosen wavelength interval; higher resolution yields finer detail but may increase computation time. The wavelength range of the simulation is set from 200 nm to 1000 nm using \texttt{low\_w=200} and \texttt{upper\_w=1000}. The maximum ionization charge state considered for spectral lines is limited by \texttt{max\_ion\_charge=3}. \texttt{max\_ion\_charge} parameter restricts the calculation to a maximum ionization stage (e.g., 2+, 3+), which can be useful to reduce computational load or to exclude highly ionized species unlikely to occur under given plasma conditions.Finally, \texttt{webscraping='static'} indicates that pre-cached atomic data is used for the simulation, avoiding live data scraping.
		
		The simulated LIBS spectrum can be visualized directly using the \texttt{plot} method. For example, the command
		
		\vspace{-0.5em}
		\begin{verbatim}
			libs.plot(color='blue', title='W Fe Mo composition')
		\end{verbatim}
		\vspace{-1.5em}
		plots the spectrum in blue and assigns the title ``W Fe Mo composition'' to the figure. Figure \ref*{fig:libs-spectrum-FeWMo} show the resultant plot.
		\begin{figure}[h!]
			\centering
			\includegraphics[width=0.8\textwidth]{FeWMo.pdf}
			\caption{LIBS spectrum for simulated sample.}
			\label{fig:libs-spectrum-FeWMo}
		\end{figure}
		
		The \texttt{simLIBS} package provides several key methods, in addition to the \texttt{plot(color, title)} method, for working with simulated spectra. 
		The \texttt{get\_raw\_spectrum()} method returns the raw spectral data obtained from the NIST atomic 
		databases, whereas the \texttt{get\_interpolated\_spectrum()} method generates a cubic spline--interpolated 
		version of the spectrum for smoother visualization and analysis. 
		For exporting results, the \texttt{save\_to\_csv(filename)} method enables saving the spectral data 
		in CSV format for external processing. 
		Finally, the \texttt{plot\_ion\_spectra()} method facilitates the visualization of spectra corresponding 
		to different ionization stages of the elements included in the simulation.
		
	\section{Simulation results}
	In this section, several simulations using the \texttt{simLIBS} package are conducted to investigate the effects of variations in elemental concentrations and plasma conditions. For the first experiment, two composition sets were selected to represent typical values found in industrial AISI~4340 samples. These two sets, referred to as the \emph{minimum} and \emph{maximum} sets, contain concentrations according to the allowable range of the AISI~4340 grade. The minimum set contains the lower limits of each constituent element, while the maximum set contains the upper limits. For both sets, the iron content was calculated as the remainder required to reach a total composition of approximately $100\%$. In addition to iron, eight elements commonly found in the alloy were considered: carbon (C), manganese (Mn), molybdenum (Mo), phosphorus (P), silicon (Si), sulfur (S), nickel (Ni), and chromium (Cr). Table \ref{tab:astm4340} show the concentration ranges of the alloying elements.
	\begin{table}[htbp]
		\centering
		\caption{ASTM 4340 Steel – Chemical Composition}
		\label{tab:astm4340}
		\resizebox{\textwidth}{!}{%
			\begin{tabular}{@{} llcccccccc @{}} 
				\toprule
				\textbf{Standard} & \textbf{Grade} & \textbf{C (\%)} & \textbf{Mn (\%)} & \textbf{P (\%)} & \textbf{S (\%)} & \textbf{Si (\%)} & \textbf{Ni (\%)} & \textbf{Cr (\%)} & \textbf{Mo (\%)} \\
				\midrule
				ASTM A29 & 4340 & 0.38--0.43 & 0.60--0.80 & $\leq$0.035 & $\leq$0.040 & 0.15--0.35 & 1.65--2.00 & 0.70--0.90 & 0.20--0.30 \\
				\bottomrule
			\end{tabular}%
		}
	\end{table}
	
	
	LIBS spectra were generated for both composition sets under identical plasma parameters, with the electron temperature fixed at 1.0~eV and the electron density at $1 \times 10^{17}~\text{cm}^{-3}$. A spectral range from 200 to 1000~nm was used, with a resolution of 1000. Ionization states up to the third charge were included in the modeling to accurately account for the plasma emission characteristics.
	
	The resulting spectra plotted together for both concentration sets is depicted in Figure \ref{fig:min-max}.
	
	
	
	\begin{figure}[h!]
		\centering
		\includegraphics[width=0.8\textwidth]{min-max.pdf}
		\caption{Simulated LIBS spectra of AISI 4240 with different sompositions (min and max ranges).}
		\label{fig:min-max}
	\end{figure} 
	The spectra corresponding to the different composition sets appear to be nearly indistinguishable, effectively overlapping one another. This observation indicates that variations in elemental concentrations within the predefined compositional ranges for AISI 4240 do not significantly affect the overall LIBS spectral profile. In other words, small to moderate changes in alloy composition within these limits result in LIBS spectra that are largely similar.
	
	For the next experiment, under the same consideration as earlier experiment, the spectra for two different steel grades, AISI 4020, and AISI 4140 were compared (See Figure \ref{fig:4020-4140}) and the results show that in this case also no significant differences were observed. suprisingly, the results spectra are very similar to thoses from earlier experiment.
	\begin{figure}[h!]
		\centering
		\includegraphics[width=0.8\textwidth]{4020-4140.pdf}
		\caption{Simulated LIBS spectra of AISI 4020 vs 4140.}
		\label{fig:4020-4140}
	\end{figure} 

	
	

	
	\section{Dataset Generation}
	The package provides a \texttt{create\_dataset} method to generate large synthetic datasets from a composition table, varying plasma parameters (\texttt{Te} and \texttt{Ne}) within user-defined ranges.
	
	Example input table:
	
	\begin{center}
		\begin{tabular}{cccc}
			\toprule
			W & H & He & name \\
			\midrule
			50 & 25 & 25 & A \\
			30 & 60 & 10 & B \\
			40 & 40 & 20 & C \\
			\bottomrule
		\end{tabular}
	\end{center}
	
	\section{Educational and Visualization Tools}
	simLIBS supports animated visualizations of how LIBS spectra change with:
	\begin{itemize}
		\item Spectral resolution,
		\item Electron temperature (\texttt{Te}),
		\item Electron density (\texttt{Ne}).
	\end{itemize}
	
	\section{References and Applications}
	\begin{itemize}
		\item Developed by Marcin Kastek (IPPLM).
		\item Applied in LIBS research:
		\begin{itemize}
			\item M. Kastek et al., \textit{Spectrochimica Acta Part B} (2022).
			\item Chen et al., \textit{Optics Letters} (2023).
		\end{itemize}
	\end{itemize}
	
	\section{Summary Table}
	\begin{center}
		\begin{tabular}{@{}ll@{}}
			\toprule
			\textbf{Feature} & \textbf{Description} \\
			\midrule
			Purpose & Simulate LIBS spectra using NIST data \\
			Install & \verb|pip install simLIBS| \\
			Inputs & Te, Ne, elements, percentages, wavelength range, resolution \\
			Outputs & Plots, raw/interpolated spectra, CSV files \\
			Dataset Tool & \verb|create_dataset| for bulk generation \\
			Educational Use & Animations for teaching \\
			License & MIT, Python $\geq$ 3.8 \\
			\bottomrule
		\end{tabular}
	\end{center}
	
		% ---------- References ----------
	\newpage
	%	\bibliographystyle{plainnat}
	\bibliographystyle{unsrtnat}
	%	\bibliographystyle{abbrvnat}
	\bibliography{references}  % Create a file named references.bib
	

	
\end{document}
