%\documentclass[12pt,a4paper]{report}
\documentclass[12pt,a4paper]{article}

% ---------- Packages ----------
\usepackage[utf8]{inputenc}
\usepackage{booktabs}
\usepackage[T1]{fontenc}
\usepackage{lmodern}
\usepackage{graphicx}
\usepackage{amsmath, amssymb}
\usepackage{geometry}
%\usepackage{hyperref}    % this keeps all colors and boxes
\usepackage{fancyhdr}
\usepackage{setspace}
\usepackage{caption}
\usepackage[numbers]{natbib}  % For bibliography
\usepackage{tocloft} % Custom TOC
\usepackage{url} 
\usepackage{amsthm}
\usepackage[most]{tcolorbox}
\usepackage{adjustbox}
\newtheorem{lemma}{Lemma}
\tcbuselibrary{theorems}
\usepackage[hidelinks]{hyperref}  % his removes all colors and boxes 
\usepackage{siunitx}    % for decimal alignment
\usepackage{subcaption} % for subfigures




\newtcbtheorem[auto counter, number within=section]{boxedlemma}{Lemma}%
{colback=blue!5!white, colframe=blue!75!black,
	fonttitle=\bfseries, coltitle=black, boxed title style={colback=white},
	enhanced, attach boxed title to top left={xshift=0.5cm,yshift=-1mm}}%
{lem}





% ---------- Page Setup ----------
\geometry{margin=1in}
\setstretch{1.2}
\pagestyle{fancy}
\fancyhf{}
\rhead{\thepage}
\lhead{A report on Real LIBS spectrum from LSA}

% ---------- Title ----------
\title{A report on Real LIBS spectrum from LSA }
\author{Ehsan Namjoo}   % Print author's name
%\date{\today}

% ---------- Document ----------{\LARGE {\tiny {\small }}}
\begin{document}
	
	\maketitle
	\tableofcontents
	\newpage
	
	
	% ---------- Sections ----------
	\section{General Overview} \label{h:intro}
	This report presents a  analysis of real LIBS data provided by the LSA. The study begins with a detailed visualization of the spectra and an examination of their intrinsic characteristics, including peak intensities and corresponding emission wavelengths. These experimental spectra are subsequently compared with those generated using the NIST LIBS simulation tool to assess consistency and spectral accuracy. Following this comparative analysis, two machine learning models—K-Nearest Neighbors (KNN) and Support Vector Machine (SVM) with an RBF kernel—are applied to classify the samples. The report concludes with  questions that need further investigation and clarification.
	
	The data is provided in two folders: One folder contains the LIBS readings for 11 standard calibration grades (samples).  For each sample there are 5–6 LIBS readings. The Table \ref{tab:Specifications for calibration grades} shows the specifications provided by the LSA for the calibration grades. The other folder contains the unlabeled LIBS Specra for random unknown materials. 
	

\begin{table}[ht]
	
	\centering
	\caption{Specifications for calibration grades.}
	\label{tab:Specifications for calibration grades}
	\begin{adjustbox}{width=\textwidth}
		\begin{tabular}{
				c
				S[table-format=2.3]
				S[table-format=2.3]
				S[table-format=2.3]
				S[table-format=2.3]
				S[table-format=2.3]
				S[table-format=2.2]
				S[table-format=2.3]
				S[table-format=2.3]
				S[table-format=2.3]
				S[table-format=2.3]
				S[table-format=3.1]
				S[table-format=2.2]
				S[table-format=3.1]
				S[table-format=2.3]
			}
			\toprule
			\textbf{No.} & {C (\%)} & {Si (\%)} & {Mn (\%)} & {P (\%)} & {S (\%)} & {Cr (\%)} & {Mo (\%)} & {Ni (\%)} & {Sn (\%)} & {Pb (\%)} & {Zn (\%)} & {Cu (\%)} & {Fe (\%)} & {Al (\%)} \\
			\midrule
			469 & 0.279 & 0.421 & 0.598 & 0.015 & 0.020 & 11.93 & {} & 0.246 & {} & {} & {} & {} & {} & {} \\
			470 & 0.153 & 0.335 & 0.235 & 0.024 & 0.035 & 17.68 & {} & 0.369 & {} & {} & {} & {} & {} & {} \\
			471 & 0.095 & 0.326 & 0.417 & 0.018 & 0.023 & 23.85 & {} & 0.960 & {} & {} & {} & {} & {} & {} \\
			472 & 0.227 & 1.050 & 1.020 & 0.032 & 0.029 & 15.82 & 0.661 & 1.950 & {} & {} & {} & {} & {} & {} \\
			473 & 0.172 & 0.604 & 0.494 & 0.019 & 0.030 & 9.06  & 0.950 & {} & {} & {} & {} & {} & {} & {} \\
			551 & {} & 0.018 & {} & 1.01 & {} & {} & {} & {0.76-} & 8.92 & 0.79 & 0.74 & 87.4 & {0.20-} & 0.052 \\
			552 & {} & 0.019 & {} & 0.77 & {} & {} & {} & {0.56-} & 9.78 & 0.63 & 0.35 & 87.7 & {0.10-} & 0.023 \\
			553 & {} & 0.022 & {} & 0.68 & {} & {} & {} & {0.44-} & {10.8-} & 0.47 & 0.49 & 87.0 & 0.056 & 0.017 \\
			554 & {} & 0.038 & {} & 0.41 & {} & {} & {} & {0.22-} & {11.3-} & 0.34 & 0.22 & 87.4 & 0.022 & 0.005 \\
			555 & {} & 0.036 & {} & 0.18 & {} & {} & {} & {0.11-} & {12.1-} & 0.24 & 0.16 & 87.1 & 0.010 & {<0.005} \\
			556 & {} & {<0.005} & {} & 0.10 & {} & {} & {} & 0.014 & {13.2-} & 0.16 & 0.09 & 86.4 & 0.004 & {<0.005} \\
			\bottomrule
		\end{tabular}
	\end{adjustbox}
	
\end{table}
Figure \ref{fig: 469} presents all LIBS readings for standard grade 469 in a single plot. From this figure, it appears that some of the readings are saturated. To allow for clearer observation, all individual readings, along with the corresponding NIST LIBS spectrum for the same sample, are plotted separately in Figure \ref{fig:469-libs-NIST}.

	\begin{figure}[h!]
	\centering
	\includegraphics[width=0.7\textwidth]{469_LSA_ava_S20251008173111_E20251008173212.png}
	\caption{All five readings of LIBS spectra for sample 469}
	\label{fig: 469}
\end{figure}
Based on the observations from Figure \ref{fig: 469} and Figure \ref{fig:469-libs-NIST}, it is evident that the spectral structure and observed emission patterns for the same element (in this case, sample 469) differ significantly from the reference spectra provided by the NIST LIBS database. This discrepancy suggests that the intensities captured using a practical, experimental LIBS device can deviate substantially from those generated synthetically using theoretical models based on the Saha–Boltzmann equations. Such differences may arise from various experimental factors, including instrumental sensitivity or plasma temperature fluctuations.
\begin{figure}[htbp]
	\centering
	\begin{adjustbox}{max width=\textwidth}
		\begin{tabular}{cc}
			\subcaptionbox{Reading 1\label{fig:469-reading1}}{\includegraphics[width=0.48\textwidth]{469-reading1.png}} &
			\subcaptionbox{Reading 2\label{fig:469-reading2}}{\includegraphics[width=0.48\textwidth]{469-reading2.png}} \\[1ex]
			\subcaptionbox{Reading 3\label{fig:469-reading3}}{\includegraphics[width=0.48\textwidth]{469-reading3.png}} &
			\subcaptionbox{Reading 4\label{fig:469-reading4}}{\includegraphics[width=0.48\textwidth]{469-reading4.png}} \\[1ex]
			\subcaptionbox{Reading 5\label{fig:469-reading5}}{\includegraphics[width=0.48\textwidth]{469-reading5.png}} &
			\subcaptionbox{NIST LIBS\label{fig:469-NIST}}{\includegraphics[width=0.48\textwidth]{469-NIST.png}} \\
		\end{tabular}
	\end{adjustbox}
	\caption{Comparison of LSA LIBS spectra with NIST for the sample 469.}
	\label{fig:469-libs-NIST}
\end{figure}

	
\section{Wavelengths and intensities} \label{h:Wavelengths and intensities}
As mentioned in an earlier section, the intensities of the LSA spectra differ significantly from those obtained from the NIST LIBS dataset. Compared to the NIST spectra, the LSA intensities appear distorted and, in some cases, saturated. This discrepancy could pose challenges in developing a model that yields consistent results across both synthetic data (from the NIST LIBS tool) and real spectra.

Regarding the wavelengths, it was observed that the sampling interval for LSA data is not constant. The intervals range from 0.035 to 0.047 nm. Figure \ref{fig:Interval_histogram} shows the histogram of these intervals, while Figure \ref{fig:Interval_distribution.png} illustrates the distribution of sampling intervals across the entire spectral range, in this case from 246.018 nm to 414.849 nm. These observations indicate that the sampling rate is smaller at the beginning of the wavelength range and increases toward the end.

	\begin{figure}[h!]
	\centering
	\includegraphics[width=0.7\textwidth]{Interval_histogram.png}
	\caption{Histogram of sampling intervals}
	\label{fig:Interval_histogram}
\end{figure}

	\begin{figure}[h!]
	\centering
	\includegraphics[width=0.7\textwidth]{Interval_distribution.png}
	\caption{Interval distribution}
	\label{fig:Interval_distribution.png}
\end{figure}

\section{Visualization}
Laser-Induced Breakdown Spectroscopy produces high-dimensional spectral data characterized by complex, nonlinear relationships between emission intensities and elemental compositions. Classical dimensionality reduction techniques such as Principal Component Analysis (PCA) are widely used for data visualization and feature extraction. However, PCA is a linear method, which assumes that the variance in the data can be captured through orthogonal transformations. This assumption often fails to represent the nonlinear and manifold-like structures inherent in LIBS spectra, where subtle variations in plasma conditions can produce nonlinear shifts in spectral intensities. 
In this section, to overcome these limitations and provide a meaningful visualization of LSA data, t-Distributed Stochastic Neighbor Embedding (t-SNE) is employed as a nonlinear dimensionality reduction technique capable of preserving local relationships between samples in the high-dimensional spectral space. Unlike PCA, which emphasizes global variance, t-SNE focuses on maintaining the probabilistic similarity of neighboring data points, making it highly effective for visualizing clusters and class separability in complex, high-dimensional datasets such as LIBS spectra.

As mentioned earlier, for every standard grade provided by LSA (see Table \ref{tab:Specifications for calibration grades}), there are 5 reading of LIBS spectra, 11 classes in total. To improve visualization stability and obtain a denser data distribution, Gaussian noise was added to the spectra to augment the dataset. This approach helps to simulate realistic measurement variability and enhances the representativeness of the spectral manifold, thereby allowing t-SNE to more effectively capture the intrinsic structure and class separability of the LIBS data. 

	\begin{figure}[h!]
	\centering
	\includegraphics[width=0.7\textwidth]{tsne.png}
	\caption{Visualization of LIBS spectra of standard grades}
	\label{fig:tsne}
\end{figure}

As observed in Figure \ref{fig:tsne}, the t-SNE visualization reduces high-dimensional spectral data to two dimensions (t-SNE 1 and t-SNE 2), allowing us to observe relationships and similarities between samples. In the plot, each patch represents a single spectrum, and colors correspond to different class labels (e.g., 469, 470, … 556). From the visualization, it is apparent that samples from each class are widely scattered rather than forming compact clusters. There is no clear group structure in which all members of a single class are positioned close to one another.

This pattern suggests poor class separability in the current feature space. Since t-SNE is designed to reveal structure in data, the lack of distinct clusters often indicates that the classes are not easily distinguishable based on these features alone. Consequently, classification using these features could be challenging, as models may struggle to find reliable decision boundaries. However, this result does not entirely rule out the possibility of successful classification. Because t-SNE is non-linear and unsupervised, distances may be distorted, and models that leverage high-dimensional decision boundaries—such as neural networks or SVMs, might still achieve separation.

Recommended next steps include quantitatively assessing separability by training simple classifiers (e.g., KNN or SVM) and evaluating their cross-validation performance. 

\section{Classification}
In this section, Support Vector Machine (SVM) and K-Nearest Neighbors (KNN) algorithms were employed to perform a multi-class classification task on the spectral dataset. The class labels were first encoded as integer values. Data augmentation was performed by adding Gaussian perturbations to each spectrum. In total, 22,456 samples were generated by perturbing the original LIBS spectra with zero-mean Gaussian noise having a variance equal to 0.7\% of the maximum peak intensity of each spectrum. The augmented dataset was then divided into training and test subsets. Finally, the models were trained on training set and evaluated on the test set. Feature scaling was applied to standardize the data prior to model training. The (0.2,0.8) split was used for test/train.

For the KNN classifier (k=15), the model achieved a very poor performance. Examination of the classification report shows that some classes (e.g., 469, 470, 471, 472, 473) were classified relatively well, whereas several classes (e.g., 551, 552, 553, 554, 555, 556) were poorly predicted, indicating that KNN struggled with certain class boundaries. This suggests that simple distance-based neighbors may not adequately capture the complex relationships between some spectral classes.

In contrast, the SVM with RBF kernel achieved near-perfect performance. Most classes were classified with perfect precision and recall, with only minor mis-classification for the class 552. This demonstrates that SVM with a non-linear kernel is highly effective at capturing the complex patterns in the high-dimensional spectral feature space, providing strong class separability. Figures \ref{fig:KNN-Confusion-Mat}, and \ref{fig:SVM-Confusion-Mat} show confusion matrices for the KNN and SVM models respectively. 

	\begin{figure}[h!]
	\centering
	\includegraphics[width=0.7\textwidth]{KNN-confusion-Matrix.png}
	\caption{Confusion matrix of the KNN model}
	\label{fig:KNN-Confusion-Mat}
\end{figure}



\begin{figure}[h!]
	\centering
	\includegraphics[width=0.7\textwidth]{SVM-Confusion-Mat.png}
	\caption{Confusion matrix of the SVM model}
	\label{fig:SVM-Confusion-Mat}
\end{figure}

\section{Conclusion}
This report aimed at investigating the characteristics of real LIBS spectrum for standard grades provided by the LSA. According to the observations and analysis performed on the data, here are some consideration and questions that require clarification.

\begin{enumerate}
	\item There are significant discrepancies between the real LIBS data and the synthetic spectra generated from the NIST LIBS dataset. These differences may pose challenges in developing reliable machine learning (ML) models that perform consistently across both real and simulated data.
	
	\item Some spectra appear to exhibit saturation effects. It is important to determine whether this saturation impacts the reliability and accuracy of the LIBS measurements. Additionally, it would be valuable to know if there are methods to minimize or prevent such saturation during data acquisition.
	
	\item The sampling intervals vary among the measurements, yet the overall spectral patterns remain similar across all readings. It is essential to understand whether these patterns are expected to remain consistent in future measurements or if they may vary under different experimental conditions.
	
	\item It was observed that simple distance-based classifiers, such as \textit{K-Nearest Neighbors (KNN)}, perform poorly in classifying standard grades, whereas more advanced models, such as \textit{Support Vector Machines (SVM)} with an \textit{RBF kernel}, achieve significantly better results. Therefore, understanding the computational and storage limitations of the target hardware is crucial for designing an efficient, fast, and reliable model.
	
	\item Each sample was measured five times. It is important to clarify whether plasma conditions remain relatively stable between consecutive measurements of the same sample, or if potential temporal variations should be considered in the analysis.
\end{enumerate}




 

%	
%	% ---------- References ----------
%	\newpage
%%	\bibliographystyle{plainnat}
%	\bibliographystyle{unsrtnat}
%%	\bibliographystyle{abbrvnat}
%	\bibliography{references}  % Create a file named references.bib
%%	
%%	% ---------- Appendix ----------
%%	\appendix
%%	\chapter{Additional Material}
%%	Include supplementary information here.
	
\end{document}
